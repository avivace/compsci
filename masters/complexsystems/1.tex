
\title{Sistemi Complessi}
\author{
		Antonio Vivace
}
\date{\today}

\documentclass[12pt]{article}
\usepackage{ dsfont }
\usepackage{amsmath}

\begin{document}
\maketitle

\section{Automi Cellulari}

\paragraph{Notazione} $A^\mathds{Z} = \{ x | x: \mathds{Z} \rightarrow A\}$
$$ f: A^\mathds{Z} \rightarrow A^\mathds{Z} $$
Richiediamo uno spazio delle configurazioni infinito $A^\mathds{Z}$ e non, ad esempio, $A^n$ altrimenti $f$ sarebbe ciclica(o comunque limitata), e non potrebbero verificarsi proprietà come l'\textit{instabilità}.
\subsection{Distanza}
Una \textit{distanza}, tra due elementi dell'insieme $X$, è una qualunque funzione $d: X \times X \rightarrow \mathds{R}^{+}$ tale che

\begin{enumerate}
	\item $d(x,y) = 0 \Leftrightarrow x = y$
	\item $d(x,y) = d(y,x) $
	\item $d(x,y) \leq d(x,z) + d(z,y)$
\end{enumerate}

Ad esempio:

$$d(x,y) = 0 \text{ se } x = y, \frac{1}{2^n} \text{ altrimenti.}$$

Dove
$$n = min\{ i \in \mathds{N} | x_i \neq y_i \vee x_{-i} \neq y_{-i} \} $$ $i$ è la larghezza della finestra che allarghiamo simmetricamente alla ricerca del primo valore diverso.

\paragraph{Notazione} Sia $x \in A^2, a,b \in \mathds{Z}, a \leq b$:
$$x[a,b] = x_a, x_{a+1}, .. x_{b} \in A^{b-a+1}
$$

\paragraph{Proposizione} Vicinanza. $\forall x,y \in A^\mathds{Z}, \forall n \in \mathds{N}$
$$d(x,y) < \frac{1}{2^n} \iff x[-n,n] = y[-n,n]$$


\subsection{Definizione formale}

\end{document}